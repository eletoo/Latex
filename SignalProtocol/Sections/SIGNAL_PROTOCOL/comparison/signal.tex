\begin{frame}{Signal VS WhatsApp VS Telegram}
    \framesubtitle{Signal}
    
    \textbf{Pro}
    \begin{itemize}
        \item Team indipendente no profit: non vende dati utente per scopi di marketing
        \item Open-source
        \item Interfaccia personalizzabile
        \item Crittografia end-to-end (anche sui metadati)
        \item Implementa il protocollo Signal
    \end{itemize}\pause

    \textbf{Contro}
    \begin{itemize}
        \item Limite dimensioni file a 100MB
    \end{itemize}

    \note{
        Informazioni raccolte: numero di telefono\\Sono in via di sviluppo nuove versioni dell'applicazione che non lo richiedano. 
    }
\end{frame}

\begin{frame}{Signal VS WhatsApp VS Telegram}
    \framesubtitle{Signal}
    L'applicazione Signal implementa il protocollo omonimo basandosi sulle librerie open-source:
    \href{https://github.com/signalapp}{GitHub - Signal App}\newline\pause
    Le versioni per iOS e Android sono strutturate su più livelli:
    \begin{itemize}
        \item Cryptographic functions layer
        \item Protocol library layer
        \item Service layer
    \end{itemize}\pause
    La versione desktop può essere utilizzata solo in associazione a un dispositivo Android o iOS.

    \note{
        Dal livello più basso al più alto: 
        \begin{itemize}
            \item Cryptographic functions layer: implementazione delle funzioni crittografiche
            \item Protocol library layer: usa le funzioni del livello crittografico per implementare il protocollo
            \item Service layer: combina le funzioni del livello protocollo per consentire di intraprendere effettivamente le conversazioni. 
        \end{itemize}

        VERSIONE DESKTOP: \newline
        All'installazione l'applicazione desktop genera una coppia di chiavi: la chiave pubblica viene presentata come QR code da scannerizzare con l'applicazione mobile. Lo smartphone crittografa l'identity key con la chiave pubblica del client desktop e la comunica al server Signal. \\
        Da questo momento l'applicazione desktop può essere usata anche se quella mobile non è in funzione.\newline

        \cite{VanDam}
    }
\end{frame}