\subsection{Proprietà}

\begin{frame}{Signal Protocol}
    \framesubtitle{Proprietà}
    Tradizionalmente facciamo riferimento a tre proprietà come requisiti principali in ambito crittografico:
    \begin{itemize}
        \item \textbf{Confidenzialità}
        \item \textbf{Integrità}
        \item \textbf{Autenticità}
    \end{itemize}\pause

    Le proprietà CIA spesso vengono accompagnate dalla \textbf{non ripudiabilità}

    \note{
        \begin{itemize}
            \item \textbf{Confidenzialità}: dati trasmessi non vengono diffusi a terzi non coinvolti nella conversazione
            \item \textbf{Integrità}: dati trasmessi non danneggiati e/o dispersi
            \item \textbf{Autenticità}: possesso di una chiave da parte di due persone al fine di riconoscere e verificare l'identità dell'altro
            \item \textbf{Non ripudiabilità}: non deve essere possibile negare per es. la propria firma a un documento, per questo spesso è implementata tramite firme digitali
        \end{itemize}
    }
\end{frame}

\begin{frame}{Signal Protocol}
    \framesubtitle{Proprietà}
    Ulteriori proprietà spesso richieste sono:
    \begin{itemize}
        \item \textbf{Forward Secrecy}: se una chiave è compromessa solo un messaggio è compromesso e non lo sono i precedenti \pause
        \item \textbf{Future Secrecy}: se una chiave è compromessa solo un messaggio è compromesso e non lo sono i successivi \pause
        \item \textbf{Cryptographic Deniability}: l'esistenza di un file cifrato o di un messaggio è rinnegabile, nel senso che un altro utente non può dimostrare che i dati in \textit{plaintext} esistono. Gli utenti possono negare che dei dati siano cifrati o anche negare di essere in grado di decifrarli, indipendentemente dal fatto che ciò sia vero o meno.
    \end{itemize}

    \note{
        Cryptographic deniability in genere è più richiesta nelle applicazioni di messaggistica 
    }
\end{frame}

%potrebbero non essere finite qui le proprietà richieste specificamente a Signal!!!