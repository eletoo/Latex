\subsection{Difetti di progettazione}

\begin{frame}{Difetti di progettazione}
    \framesubtitle{Frosch analysis \cite{7467371}}
    \begin{itemize}
        \item \textit{Export function} [fixed]: la funzione esportava la password necessaria per inviare le \textit{one-time pre-keys} al server in \textit{plaintext} su dispositivi Android
        \item \textit{Vulnerabilità a UKS - Unknown key-share attack}
    \end{itemize}

    \note{
        I difetti di progettazione individuati in questo paper sono stati ricavati da un'analisi di X3DH e di Double Ratchet separatamente, ma non da un'analisi dell'interazione dei due algoritmi come invece avviene nel protocollo Signal.\newline

        UKS\\
        Esempio: \\Bob sa che Charlie lo inviterà a una festa. Per fare uno scherzo a Charlie, Bob sostituisce la propria chiave con quella di Dave.\\
        Quando Charlie invita Bob alla festa, Bob inoltrerà il messaggio a Dave.\\
        Dal punto di vista di Dave, sembrerà che Charlie abbia inviato il messaggio.\\
        Charlie penserà di aver invitato Bob, ma avrà in effetti invitato Dave.
    }
\end{frame}

\begin{frame}{Difetti di progettazione}
    \framesubtitle{Cohn-Gordon analysis \cite{7961996}}

    L'analisi effettuata è valida presupponendo che tutte le KDF si comportino come oracoli che restituiscono un output simile a un output casuale.\newline\pause

    Definiamo questa condizione come \textit{random oracle model}.\newline

    \note{
        Il paper \cite{7961996} è il primo studio scientifico a riportare un'analisi formale dell'interazione tra X3DH e Double Ratchet.
    }
\end{frame}

\begin{frame}{Difetti di progettazione}
    \framesubtitle{Cohn-Gordon analysis \cite{7961996}}
    
    L'attacco UKS individuato in \cite{7467371} non è valido in questo modello perché esso, come in effetti anche il protocollo Signal, non differenzia gli identificatori di sessione (i.e. l'ID della sessione Alice-Bob è lo stesso ID della sessione Alice-Charlie). \linebreak\pause

    Non sono stati individuati altri rilevanti difetti di progettazione che generino punti di debolezza nel protocollo.

    \note{
        UKS è un attacco prevenibile a livello di applicazione (introducendo degli identificativi per gli utenti, e.g. il numero di telefono).\\
        N.B. Gli attacchi possono variare a seconda dell'implementazione fornita del protocollo, ma il protocollo in sé non presenta difetti rilevanti.
    }
\end{frame}
