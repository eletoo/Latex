\begin{frame}{Signal Protocol}
    \framesubtitle{Il protocollo: Sesame}

    L'algoritmo \textbf{Sesame} gestisce la creazione, eliminazione e utilizzo delle sessioni di comunicazione.\newline\pause
    Ogni dispositivo degli utenti deve tenere traccia di una sessione \textit{attiva} per ogni altro dispositivo con cui sta comunicando e utilizzare quella sessione quando comunica con esso.
    Quando si riceve un messaggio su una sessione \textit{inattiva}, essa diventa \textit{attiva}.\pause\newline
    In questo modo ogni dispositivo utilizza una sola sessione per ogni altro dispositivo remoto con cui comunica. 
    \cite{sesame}

    \note{
        Problemi che si possono generare:
        \begin{itemize}
            \item Alice e Bob possono avere ognuno più dispositivi quindi l'invio di un messaggio da parte di Alice richiede di inviarne una copia a tutti i dispositivi di Bob e a tutti i dispositivi di Alice.
            \item Alice e Bob possono aggiungere o rimuovere dispositivi, iniziando o terminando delle sessioni.
            \item Alice e Bob possono iniziare una nuova sessione allo stesso tempo. Per il funzionamento del Double Ratchet Alice e Bob devono usare la stessa sessione, quindi devono concordare su quella da utilizzare.
            \item Alice può resettare lo stato della sessione sul suo dispositivo, richiedendo a Bob di concordare nuovamente sulla sessione da utilizzare.
        \end{itemize}
    }
\end{frame}

\begin{frame}{Signal Protocol}
    \framesubtitle{Il protocollo: Sesame}
    Sesame è stato progettato per l'uso attraverso sessioni Double Ratchet create attraverso scambio di chiavi X3DH.

    \begin{itemize}
        \item I dispositivi comunicano al server le proprie \textit{one-time pre-keys}, \textit{signed pre-keys} e \textit{identity public key}.\pause
        \item Il dispositivo mittente recupera dal server la \textit{identity public key} del dispositivo destinatario, \textit{signed pre-keys} e una \textit{one-time pre-key} se disponibile.\pause
        \item X3DH usa queste chiavi per creare sia una chiave segreta che apre una sessione Double Ratchet sia un messaggio iniziale X3DH.
    \end{itemize}    

\end{frame}

\begin{frame}{Signal Protocol}
    \framesubtitle{Il protocollo: Sesame}
    \begin{itemize}        
        \item Il messaggio iniziale X3DH è aggiunto a ogni messaggio di apertura di sessione, in modo che il destinatario lo usi per creare una sessione Double Ratchet corrispondente.\pause
        \item Ricevuto il messaggio di conferma di apertura della sessione il mittente smette di inviare il messaggio iniziale X3DH. \pause
        \item I dispositivi comunicano solo con Double Ratchet.
    \end{itemize}

    \cite{sesame}, \cite{VanDam}
\end{frame}