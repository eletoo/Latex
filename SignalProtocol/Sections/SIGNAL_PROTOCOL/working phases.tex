\begin{frame}{Signal Protocol}
    \framesubtitle{Il protocollo: fasi di funzionamento \cite{VanDam}}
    \begin{itemize}
        \item KEY REGISTRATION: invio di numerose chiavi pubbliche al server per consentire di iniziare una conversazione mentre l'altro utente non è online\pause
        \item KEY AGREEMENT: Alice riceve le chiavi pubbliche di Bob dal server e le usa, insieme alle proprie chiavi private, per generare una chiave segreta condivisa. Invia a Bob un messaggio criptato con questa chiave. Bob, ricevutolo, recupera le chiavi pubbliche di Alice dal server e calcola la stessa chiave segreta condivisa. La negoziazione delle chiavi avviene tramite X3DH\pause
        \item CONVERSATION: Alice e Bob possiedono la chiave segreta condivisa e possono conversare.
                            \begin{itemize}
                                \item DH ratchet phase
                                \item Symmetric ratchet phase                                
                            \end{itemize}
    \end{itemize}

    \note{
        \begin{itemize}
            \item KEY REGISTRATION: se Alice vuole iniziare una conversazione con Bob può chiedere al server le sue chiavi pubbliche
            \item KEY AGREEMENT
            \item CONVERSATION 
        \end{itemize}

        N.B. Symmetric ratchet phase e DH ratchet phase verranno meglio analizzati più avanti nel parlare dell'algoritmo Double Ratchet.  
    }
\end{frame}

\begin{frame}{Signal Protocol}
    \framesubtitle{Il protocollo: fasi di funzionamento}
    \textbf{Symmetric ratchet phase}\newline

    Derivazione di una nuova chiave dalla chiave segreta condivisa.\newline\pause
    Se Alice invia più messaggi a Bob senza ricevere risposta ogni messaggio sarà criptato con una nuova chiave calcolata in funzione della precedente.\newline\pause
    In questo modo solo Alice e Bob possono calcolarla (escludendo casi in cui la chiave sia compromessa)
    
\end{frame}

\begin{frame}{Signal Protocol}
    \framesubtitle{Il protocollo: fasi di funzionamento}
    \textbf{Diffie–Hellman ratchet phase}\newline

    Generazione di una nuova chiave segreta condivisa.\newline\pause
    Se Bob invia un nuovo messaggio ad Alice genera una nuova coppia di chiavi effimere. Bob usa queste chiavi per calcolarne una nuova condivisa, inviando poi la propria chiave effimera ad Alice per farle calcolare la chiave condivisa.\newline\pause
    La chiave così calcolata verrà usata in una nuova \textit{symmetric ratchet phase} per generare nuove chiavi per i messaggi.

\end{frame}