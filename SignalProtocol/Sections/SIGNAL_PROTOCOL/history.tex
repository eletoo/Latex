\subsection{Storia}

\note{
    Contrariamente alle due politiche \textit{Security by Design} e \textit{Open security}
    la sicurezza tramite offuscazione è fortemente sconsigliata, in quanto affida la sicurezza del sistema
    al fatto che nessuno riesca a comprenderlo.

    Questa pratica rende quindi il sistema vulnerabile a qualsiasi attacco di tipo reverse engeneering,
    oltre che a possibili fughe di informazioni.

    L'utilizzo di ideologie ``open'' permette la validazione del sistema da parte di un maggior numero di enti
    e di membri di una comunità, permettendo così l'individuazione di falle in minor tempo.

    Il metodo più efficiente, però, consiste sempre nell'utilizzo di sistemi già esistenti e ritenuti sicuri (p.e. tritium)
}
