\subsection{Storia dell'Applicazione}

\begin{frame}
    \frametitle{Applicazione Signal}
    \textbf{Storia dell'Applicazione}
    \newline
    
    L’applicazione Signal ha origine dall’unione dei due servizi di messaggistica \textbf{TextSecure} e \textbf{RedPhone}, sviluppati da \textbf{Moxie Marlinspike} e \textbf{Stuart Anderson}, che insieme fondarono la start-up \textbf{Whisper Systems} nel 2010.\newline \pause

    Entrambe le applicazioni implementavano la crittografia end-to-end.

\end{frame}

\begin{frame}{Applicazione Signal}

    A seguito del nuovo rilascio delle applicazioni nel 2011 i due servizi assumono la propria natura \textbf{open-source} che ancora oggi caratterizza l’applicazione Signal.\newline \pause
    
    Nel 2013 Marlinspike fonda il progetto open-source \textbf{Open Whisper Systems}, grazie a cui rilascia la prima versione di Signal nel 2015 (anche per PC come applicazione Chrome), per poi rilasciarlo anche per Windows, Mac e Linux nel 2017. \newline
    
    \note{
        Nel 2011 Twitter acquista Whisper Systems e Marlinspike diventa capo della cybersecurity del social media.\\
        Nel 2013 Marlinspike abbandona Twitter e fonda la OWS.\\
        Nello stesso anno inizia a lavorare al protocollo Signal insieme al fondatore di WhatsApp Trevor Perrin.
    }
\end{frame}

\begin{frame}{Applicazione Signal}

  Nel febbraio 2018 Marlinspike e il co-fondatore di WhatsApp Brian Acton fondarono la \textbf{Signal Foundation}, il cui obiettivo è il supporto e l’accelerazione della diffusione della comunicazione privata e sicura.
  \cite{Lumb}
    
\end{frame}