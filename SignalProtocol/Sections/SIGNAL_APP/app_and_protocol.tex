\subsection{L'Applicazione e il Protocollo Signal}

\begin{frame}{Applicazione Signal}
    \textbf{L'Applicazione e il Protocollo Signal}
    \newline
    
    Nel 2013, dopo la fondazione di OWS, i fondatori Marlinspike e Trevor Perrin iniziarono a lavorare al \textbf{Protocollo Signal}.\newline\pause
    
    Esso rendeva il metodo crittografico end-to-end utilizzato nell’applicazione Signal implementabile anche da altri servizi.\newline\pause
    
    Ogni piattaforma di messaggistica che intraprese collaborazioni con OWS al fine di integrare il protocollo Signal al proprio interno lo implementò in modalità differenti e su scala/estensione diversa.

\end{frame}

\begin{frame}{Applicazione Signal}
    Tra le più note implementazioni (parziali) del Protocollo Signal troviamo:\pause
    \begin{itemize}
        \item <1-> \textbf{Facebook}: introdusse la feature \textit{Secret Conversations} per gli utenti di Facebook Messenger nel luglio 2016\pause
        \item <2-> \textbf{Allo}: rilasciata nel settembre 2016, sfruttava il Protocollo Signal se utilizzata in modalità incognito\pause
        \item <3-> \textbf{Duo}: protezione delle videochat\pause
        \item <4-> \textbf{Skype}: conversazioni private dal 2018\pause
        \item <5-> \textbf{WhatsApp}: tra le maggiori applicazioni che implementano Signal è l’unica che garantisce di default la crittografia end-to-end delle conversazioni (da aprile 2016)
    \end{itemize}
    
    \note{
     \begin{itemize}
         \item Facebook: usa Signal solo nelle Secret Conversations
         \item Allo: applicazione mobile di messaggistica istantanea di Google, non esiste più dal 12 marzo 2019 
         \item Duo: applicazione per videochiamate e chat mobile di Google
         \item Whatsapp: introdusse Signal per la prima volta nel 2014 per utenti Android, estendendolo a tutti gli utenti nel 2016
         \item Google: introduce Signal di default nell’applicazione di messaggi su Android
     \end{itemize}
}
\end{frame}

\begin{frame}{Applicazione Signal}
    Ciascuna di queste \textit{features} richiede che le conversazioni intraprese siano dichiarate “private” affinché sia possibile applicare la crittografia end-to-end su tutto il contenuto che viene scambiato\newline\pause 

    Inoltre, conversazioni già avvenute non possono essere protette applicando il protocollo ex post.

    \note{
        La dichiarazione delle conversazioni come “private” avviene in genere per selezione esplicita da parte dell'utente e non di default.\newline
        WhatsApp implementa automaticamente la crittografia end-to-end sia per le chat private che per quelle di gruppo, tuttavia se si vuole verificare che le conversazioni siano private è necessario che entrambe le persone che partecipano alla conversazione selezionino la chat di interesse, clicchino sul nome del contatto, selezionino l'opzione “Crittografia” e scannerizzino il codice QR che viene presentato sul dispositivo dell'altro utente oppure confrontino i numeri a 60 cifre presentati.\newline
    }
\end{frame}

\begin{frame}{Applicazione Signal}
    La sicurezza garantita dall'implementazione del protocollo è relativa al fatto che tutti i prodotti OWS sono incentrati sulla privacy degli utenti, infatti:
    \begin{itemize}
        \item Salvano solo le informazioni strettamente necessarie\pause
        \item Rendono impossibile a terze parti accedere ai messaggi o ai file scambiati tra gli utenti (grazie alla crittografia end-to-end)
    \end{itemize}
\end{frame}