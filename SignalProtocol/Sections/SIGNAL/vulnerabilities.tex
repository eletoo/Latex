\subsection{Vulnerabilità}

\begin{frame}
    Le vulnerabilità trovate negli anni nel crittosistema sono le seguenti.

    Alcune di queste sono di facile soluzione perchè riguardano il lato hardware del lettore,
    per cui, a fronte di un costo maggiore, è possibile migliorarne le capacità.
\end{frame}

\begin{frame}
    \scriptsize
    \begin{itemize}
        \item <1-> Utilizzo di chiavi a 48 bit: Possibili attacchi BruteForce \cite{courtois2008algebraic}
        \item <2-> RNG del tag non è crittograficamente sicuro. Infatti è un LFSR con condizione iniziale costante.
        \begin{itemize} \scriptsize
            \item <3-> Consegue che lo stato è prevedibile e dipende dal tempo trascorso dal poweron~\cite{garcia2008dismantling}\cite{courtois2008algebraic}
            \item <4-> Inoltre i numeri casuali sono generati a partire da 16 bit del registro 
            \item <5-> in particolare i numeri sono generati ad ogni ciclo di clock del tag, quindi la precisione del attaccante deve limitarsi a quanti di 10 microsecondi (106kHz) e la sequenza di numeri i ripete ogni 65535 iterazioni (0.6s)
        \end{itemize}
        \item <6-> L'RNG dei lettori viene aggiornato solamente ad ogni nuova autenticazione~\cite{garcia2008dismantling}
        \item <7-> La funzione di filtraggio del LFSR usa 20bit del registro e sono solo bit in posizione dispari
        \item <8-> LFSR State Recovery
        \item <9-> LFSR Rollback
        \item <10-> i bit di parità sono computati sul plaintext e poi inviati non cifrati
        \item <11-> nested authentication attacks
    \end{itemize}
\end{frame}


\subsubsection{LFSR}
\begin{frame}
    \frametitle{Dettagli sul LFSR}
    todo: 
        reverse engeneering \cite{nohl2008reverse} del chip per trovare il circuito di autenticazione
        attacchi per trovare la funzione \cite{garcia2008dismantling}
        attacchi dati dai odd
        nested auth e drschottky
\end{frame}

\subsubsection{Attacchi Algebrici}
\begin{frame}
    \frametitle{Attacchi Algebrici}

\end{frame}

