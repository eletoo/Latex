\section{Applicazione Signal}

\subsection{Storia dell'Applicazione}
\begin{frame}
    \frametitle{Applicazione Signal}
    \textbf{Storia dell'Applicazione}
    \newline
    
    L’applicazione Signal ha origine dall’unione dei due servizi di messaggistica \textbf{TextSecure} e \textbf{RedPhone}, sviluppati da \textbf{Moxie Marlinspike} e \textbf{Stuart Anderson}, che insieme fondarono la start-up \textbf{Whisper Systems} nel 2010.\newline

    Entrambe le applicazioni implementavano la crittografia end-to-end.

\end{frame}

\begin{frame}{Applicazione Signal}

    A seguito del nuovo rilascio delle applicazioni nel 2011 i due servizi assumono la propria natura \textbf{open-source} che ancora oggi caratterizza l’applicazione Signal.\newline
    
    Nel 2013 Marlinspike fonda il progetto open-source \textbf{Open Whisper Systems}, grazie a cui rilascia la prima versione di Signal nel 2015 (anche per PC come applicazione Chrome), per poi rilasciarlo anche per Windows, Mac e Linux nel 2017. \newline
    
    \note{
        Nel 2011 Twitter acquista Whisper Systems e Marlinspike diventa capo della cybersecurity del social media.\\
        Nel 2013 Marlinspike abbandona Twitter e fonda la OWS.\\
        Nello stesso anno inizia a lavorare al protocollo Signal insieme al fondatore di WhatsApp Trevor Perrin.
    }
\end{frame}

\begin{frame}{Applicazione Signal}

  Nel febbraio 2018 Marlinspike e il co-fondatore di WhatsApp Brian Acton fondarono la \textbf{Signal Foundation}, il cui obiettivo è il supporto e l’accelerazione della diffusione della comunicazione privata e sicura.
    
\end{frame}

\subsection{L'Applicazione e il Protocollo Signal}
\begin{frame}{Applicazione Signal}
    \textbf{L'Applicazione e il Protocollo Signal}
    \newline
    
    Nel 2013, dopo la fondazione di OWS, i fondatori Marlinspike e Trevor Perrin iniziarono a lavorare al \textbf{Protocollo Signal}.\newline\pause
    
    Esso rendeva il metodo crittografico end-to-end utilizzato nell’applicazione Signal implementabile anche da altri servizi. \newline\pause
    
    Ogni piattaforma di messaggistica che intraprese collaborazioni con OWS al fine di integrare il protocollo Signal al proprio interno lo implementò in modalità differenti e su scala/estensione diversa.

\end{frame}

\begin{frame}{Applicazione Signal}
    Tra le più note implementazioni (parziali) del Protocollo Signal troviamo:
    \begin{itemize}
        \item <1-> \textbf{Facebook}: introdusse la feature \textit{Secret Conversations} agli utenti di Facebook Messenger nel luglio 2016
        \item <2-> \textbf{Allo}: rilasciata nel settembre 2016, utilizzava il Protocollo Signal se utilizzata in modalità incognito
        \item <3-> \textbf{Duo}: protezione delle videochat
        \item <4-> \textbf{Skype}: conversazioni private dal 2018
        \item <5-> \textbf{WhatsApp}: tra le maggiori applicazioni che implementano Signal è l’unica che garantisce di default la crittografia end-to-end delle conversazioni (da aprile 2016)
    \end{itemize}
    
    \note{
     \begin{itemize}
         \item Facebook: solo nelle Secret Conversations 
         \item Allo: applicazione di messaggistica di Google, non esiste più dal 12 marzo 2019 
         \item Whatsapp: introdotto per la prima volta nel 2014 per utenti Android, esteso a tutti gli utenti nel 2016
         \item Google: introdotto di default nell’applicazione di messaggi su Android
     \end{itemize}
}
\end{frame}

