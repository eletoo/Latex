\subsection{Applicazioni}
\begin{frame}
    \frametitle{End-to-End Encryption}
    \textbf{Applicazioni}\newline

    \begin{itemize}
        \item \textbf{Comunicazioni sicure}: applicazioni di messaggistica e posta elettronica per mantenere private le conversazioni degli utenti;
        \item \textbf{Gestione password}: in questo caso a entrambi gli endpoint della comunicazione si trova lo stesso utente, che è l'unica persona munita di chiave;
        \item \textbf{Data storage}: nei servizi di storage in cloud può anche essere garantita E2EE \textit{in transit}, proteggendo i dati degli utenti anche dall'accesso da parte dei fornitori del servizio in cloud;
    \end{itemize}

    \note{
        La protezione dei dati tramite \textit{encryption in transit} consiste nel cifrare i dati solo lungo il percorso su cui vengono
        trasmessi ma non alla sorgente. In queste condizioni, colui che invia i dati ha accesso al loro contenuto, cosa che si vuole spesso evitare.

    }
\end{frame}