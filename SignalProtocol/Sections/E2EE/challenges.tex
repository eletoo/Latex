\subsection{Problematiche}

\begin{frame}
    \frametitle{End-to-End Encryption}
    \textbf{Problematiche}\newline

    La E2EE non garantisce di per sé né la sicurezza né la privacy, in quanto i dati trasmessi potrebbero essere protetti in modo poco sicuro sui dispositivi endpoint.\newline
    Tuttavia, l'implementazione della E2EE consente di applicare una protezione dei dati migliore della sola crittografia ``in transit''.\newline\pause

    Per molti sistemi di messaggistica i messaggi passano attraverso un intermediario che li conserva finché non vengono recuperati dal destinatario. Anche se protetti da crittografia, essi lo sono solamente in transito, quindi possono essere letti dai provider di servizi. 

    \cite{intransitEncryption}, \cite{IBM}

    \note{
        In questo modo è possibile monitorare il contenuto dei messaggi (per esempio in cerca di contenuti offensivi o pericolosi) ma si corre anche il rischio che utenti non autorizzati e/o malintenzionati aventi accesso allo storage dei messaggi possano fare un uso improprio dei contenuti.\newline

        Nella crittografia ``in transit'' è possibile o salvare direttamente i messaggi decrittati oppure salvare i dati crittografati e la chiave con cui decrittarli sullo stesso database.  
    }
\end{frame}

\begin{frame}
    \frametitle{End-to-End Encryption}
    Ulteriori problematiche: 
    \begin{itemize}
        \item \textbf{Endpoint security}: gli endpoint sono vulnerabili se non protetti adeguatamente\pause
        \item \textbf{Attacchi di tipo Man-in-the-Middle}: la conversazione può essere soggetta a \textit{eavesdropping}\pause
        \item \textbf{Backdoors}: si tratta di metodi per bypassare l'autenticazione standard o la protezione crittografica di un dispositivo. Se non volute, possono essere introdotte tramite attacchi cyber e poi sfruttate per violare la sicurezza del sistema
    \end{itemize}

    \cite{greenbergE2EE}, \cite{IBM}

    \note{
        \begin{itemize}
            \item Endpoint security: E2EE protegge i dati solo tra i due endpoint; ciò significa che i due endpoint possono essere soggetti ad attacchi;
            \item Attacchi MITM: anziché forzare la crittografia dei dati, ci si può aspettare un tentativo da parte di terzi malintenzionati di impersonare il destinatario durante. Essi possono, per esempio, impersonare il destinatario durante lo scambio di chiavi con il mittente, decifrare il messaggio inviato e poi inoltrarlo al vero destinatario senza farsi notare. Una soluzione per questo tipo di attacchi è introdurre un metodo di autenticazione (per es. certification authorities, web of trust, fingerprint numeriche o come QR code)
            \item Backdoors: nonostante le \textit{backdoors} non siano sempre implementate volutamente, esse possono essere introdotte grazie a \textit{cyber-attacks} e poi essere utilizzate per
            la negoziazione delle chiavi o per oltrepassare la protezione crittografica.
        \end{itemize}
    }
\end{frame}

\begin{frame}
    \frametitle{End-to-End Encryption} 
    \begin{itemize}
        \item Complessità nel definire gli endpoint: alcune implementazioni consentono di decodificare e ricodificare i dati lungo il percorso, quindi è necessario definire accuratamente gli estremi della trasmissione\pause
        \item Privacy ``eccessiva'': enti governativi non hanno modo di verificare la natura dei contenuti trasmessi dagli utenti, pertanto non sono in grado di prendere misure adeguate in caso di illeciti\pause
        \item Metadati visibili\pause
        \item Non vi è certezza che E2EE possa funzionare altrettanto bene con l'eventuale introduzione di \textit{quantum computer} che rendano la crittografia obsoleta
    \end{itemize}
    \cite{techTarget}
\end{frame}