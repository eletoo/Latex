\subsection{Problematiche}
\begin{frame}
    \frametitle{End-to-End Encryption}
    \textbf{Problematiche}\newline

    \begin{itemize}
        \item \textbf{Endpoint security}: gli endpoint sono vulnerabili se non protetti adeguatamente
        \item \textbf{Attacchi di tipo Man-in-the-Middle}: la conversazione può essere soggetta a \textit{eavesdropping}
        \item \textbf{Backdoors}: se non volute, possono essere introdotte tramite attacchi cyber e poi sfruttate per violare la sicurezza del sistema
    \end{itemize}

    \note{
        \begin{itemize}
            \item Endpoint security: E2EE protegge i dati solo tra i due endpoint; ciò significa che i due endpoint possono essere soggetti ad attacchi;
            \item Attacchi MITM: la conversazione può essere soggetta a \textit{eavesdropping} da parte di terzi malintenzionati in grado di intercettare i messaggi
            e impersonare il destinatario. Essi possono, per esempio, scambiare le chiavi con il mittente, decifrare il messaggio inviato e poi inoltrarlo al vero destinatario senza farsi notare;
            \item Backdoors: nonostante le \textit{backdoors} non siano sempre implementate volutamente, esse possono essere introdotte grazie a \textit{cyber-attacks} e poi essere utilizzate per
            la negoziazione delle chiavi o per oltrepassare la protezione crittografica.
        \end{itemize}
    }
\end{frame}